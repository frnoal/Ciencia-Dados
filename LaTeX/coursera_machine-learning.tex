\textsf{Coursera} - Machine Learning
Professor: Geoffrey Hinton - University of Toronto
% Aluno: Filipe Ronald Noal\\\texttt{filipe.ronald@gmail.com}

\def\r          {`\textsf{R}'}%
\def\octave     {\textsf{GNU Oc\-ta\-ve}}%


\begin{abstract}
Welcome to Neural Networks for Machine Learning! You’re joining thousands of learners currently enrolled in the course. I'm excited to have you in the class and look forward to your contributions to the learning community.

To begin, I recommend taking a few minutes to explore the course site. Review the material we’ll cover each week, and preview the assignments you’ll need to complete to pass the course. Click Discussions to see forums where you can discuss the course material with fellow students taking the class.

If you have questions about course content, please post them in the forums to get help from others in the course community. For technical problems with the Coursera platform, visit the Learner Help Center.

Good luck as you get started, and I hope you enjoy the course!
\end{abstract}


%%%%%%%%%%
\section{Semana I}


%%%%%%%%%%
\subsection{Syllabus and Course Logistics}

Welcome to the course!

There are 16 modules ("lectures") and each module will be divided into about five short videos. In each module there will be a quiz that counts towards your final grade.

The modules are broken down into the following:

Lecture 1: Introduction

Lecture 2: The Perceptron learning procedure

Lecture 3: The backpropagation learning procedure

Lecture 4: Learning feature vectors for words

Lecture 5: Object recognition with neural nets

Lecture 6: Optimization: How to make the learning go faster

Lecture 7: Recurrent neural networks

Lecture 8: More recurrent neural networks

Lecture 9: Ways to make neural networks generalize better

Lecture 10: Combining multiple neural networks to improve generalization

Lecture 11: Hopfield nets and Boltzmann machines

Lecture 12: Restricted Boltzmann machines (RBMs)

Lecture 13: Stacking RBMs to make Deep Belief Nets

Lecture 14: Deep neural nets with generative pre-training

Lecture 15: Modeling hierarchical structure with neural nets

Lecture 16: Recent applications of deep neural nets (optional videos)

The two module quizzes

Each module, there will be two quizzes (one per lecture) that do count towards your final grade.

The programming assignments

You will need to answer questions about the results produced by the programs and your answers will count towards your final grade.

The first programming assignment will be very simple. It is mainly intended to get you to download Octave and get used to using it (see the Octave installation link). We regret that we do not have the resources to support other languages, but if you have Matlab it should be simple to adapt the Octave code we provide. You will not need to submit any code.

The final test

The final test will be 25\% of the final grade, the programming assignments will be 30\% and the weekly quizzes 45%.

Please remember that this course contains the same content presented on Coursera beginning in 2013. It is not a continuation or update of that original course

Help

If you have a question about using the Coursera platform, your Coursera account, or any technical issues, please visit Coursera's Learner Help Center. Coursera regularly updates the Help Center with solutions to the most common learner issues. You'll also find a learner-to-learner Support Forum community where learners can post questions or read past responses to get immediate answers. If you still need help after troubleshooting, 24/7 chat is available or you can submit a support request to Coursera at the bottom of relevant resources.


%%%%%%%%%%
\subsection{}
